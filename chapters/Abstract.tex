\begin{abstract}
Il lavoro dell'elaborato esamina metodi avanzati per ottimizzare le \textit{query} nei sistemi di \textit{Business Intelligence} (BI), con un \textit{focus} specifico sui \textit{Data Warehouse}. L'obiettivo principale è individuare strategie efficaci per migliorare le prestazioni e il supporto decisionale di tali sistemi, diventati sempre più importanti per la competitività aziendale. Partendo da un'analisi critica della letteratura scientifica, sono state esplorate tematiche come l'ottimizzazione fisica delle \textit{query} attraverso l'uso di indici, partizionamento, compressione e caching, l'applicazione strategica delle viste materializzate e l'implementazione di approcci \textit{data-driven} basati sull'apprendimento automatico.
\\ \\
La metodologia adottata comprende una revisione approfondita della letteratura tecnica, un confronto tra algoritmi euristici, genetici e ibridi per la selezione delle viste materializzate, e uno studio applicativo del sistema Kepler, una soluzione parametrica basata su reti neurali per l'ottimizzazione delle \textit{query}. \\
Inoltre, un'analisi scientometrica ha permesso di identificare i \textit{trend} evolutivi degli ultimi vent'anni e le lacune ancora presenti nella ricerca sul tema.
\\ \\
I risultati dimostrano che tecniche come le viste materializzate incrementali, gli algoritmi ibridi genetici e i metodi \textit{data-driven} possono ridurre significativamente i tempi di risposta alle \textit{query} e migliorare la scalabilità dei \textit{Data Warehouse}, con un uso più efficiente delle risorse. \\
L'analisi effettuata ha anche evidenziato come l'integrazione tra metodi tradizionali e tecnologie moderne basate sull'\textit{machine learning} possa rappresentare un fattore cruciale per l'efficienza dei sistemi di \textit{Business Intelligence}.
\\ \\
Infine, il lavoro presenta linee guida operative per la progettazione e l'implementazione pratica di soluzioni di ottimizzazione delle \textit{query} in contesti reali, delineando anche prospettive per lo sviluppo futuro di questa area.
\end{abstract}