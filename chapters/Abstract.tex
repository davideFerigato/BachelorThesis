\begin{abstract}
L’elaborato analizza tecniche avanzate per l’ottimizzazione delle query nei sistemi di \textit{Business Intelligence}, considerando architetture di \textit{Data Warehouse} ed ETL. Il lavoro integra teoria e pratica: dalla progettazione fisica (indicizzazione, partizionamento, compressione, \textit{caching}) alla materializzazione delle viste, fino a metriche e \textit{governance} misurabili delle prestazioni. L’obiettivo è definire procedure riproducibili per ridurre latenza e I/O, preservando qualità e aggiornamento dei dati.
\\ \\
La trattazione segue un percorso progressivo. Si richiamano i fondamenti di BI e le differenze tra OLTP e OLAP; si esamina la qualità degli ETL e l’impatto delle metriche strutturali; si approfondiscono le scelte di \textit{design} fisico per carichi analitici. Si inquadra il ruolo delle viste materializzate nella riduzione del costo delle interrogazioni e, nel capitolo applicativo, si analizzano SQL Server e Azure Synapse: requisiti di creazione, riscrittura automatica delle \textit{query}, distribuzione dei dati, statistiche, uso di CTAS e tabelle temporanee, strumenti di diagnostica e \textit{overhead}.
\\ \\
La metodologia combina la revisione della letteratura, un confronto tra approcci classici per selezione e manutenzione delle viste, e un protocollo sperimentale ispirato a \textit{benchmark} decisionali. Le metriche prese in considerazione includono tempi medi e dispersione, costo di manutenzione in presenza di aggiornamenti e impatto su I/O e spazio.
\\ \\
I risultati convergono in linee guida operative: base fisica \textit{columnstore} e partizionamento, viste progettate su \textit{join} e aggregazioni condivise con distribuzione coerente, validazione mediante riscrittura automatica e soglie di \textit{overhead} per il \textit{rebuild}. Nel complesso, l’adozione coordinata di tali tecniche riduce in modo stabile le latenze e migliora l’efficienza globale del \textit{Data Warehouse}, fornendo indicazioni applicabili e replicabili.

\end{abstract}